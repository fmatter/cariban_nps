\documentclass[10pt]{article}
\usepackage[
    landscape,
    twocolumn,
    a4paper,
    left=0.7in,
    right=0.7in,
    bottom=1in,
    top=0.7in]{geometry}

\usepackage{fontspec}
\setmainfont{Brill}
\usepackage[abbrevs=none,refmode=latex]{expex-acro}
\usepackage{booktabs}
\usepackage[style=authoryear]{biblatex}
\usepackage[textwidth=30mm]{todonotes}
\def\tightlist{}
\usepackage{hyperref}
\usepackage[capitalise]{cleveref}

\addbibresource{/home/florianm/Dropbox/research/cariban/papers/np_discont/data/cldf/sources.bib}
\lingset{belowglpreambleskip=-1ex,everyglpreamble=\itshape,aboveglftskip=-0.5ex,aboveexskip=-0.7ex,belowexskip=-1ex}

\newGlossingAbbrev{3}{third person}
\newGlossingAbbrev{n}{noun}
\newGlossingAbbrev{np}{noun phrase}
\newGlossingAbbrev{1+2}{1+2}
\newGlossingAbbrev{a}{a}
\newGlossingAbbrev{p}{p}
\newGlossingAbbrev{t}{t}
\newGlossingAbbrev{g}{g}
\newGlossingAbbrev{o}{o}
\newGlossingAbbrev{dem}{dem}
\newGlossingAbbrev{2}{2}
\newGlossingAbbrev{1}{1}

\setmainfont{Linux Libertine O}

\begin{document}
\begin{center}
\Large \bfseries {Apparent discontinuous NPs in the Cariban family from the perspective of diachronic typology}
\end{center}
%\tableofcontents

\section{Introduction: Panare}

\begin{itemize}
\tightlist
\item
  \textcites{payne1993nonconfigurationality} discusses discontinuous
  nominal expressions in Panare:
\end{itemize}

\pex\label{}    \a Panare\\
    \label{pan-0}        \begingl
        \glpreamble Kana tënyaj chu tityasa’. //
        \gla **kana** t-ën-yaj chu **tityasa’**//
        \glb fish \gl{1}>\gl{3}-catch-\gl{pst}.\gl{pfv} \gl{1}\gl{sg} one//
            \glft ‘I caught one fish.’//  
        \endgl 
    \a Panare\\
    \label{pan-1}        \begingl
        \glpreamble Ta’meñe nuwïyaj ana arakon. //
        \gla **ta’meñe** nu-wï-yaj ana **arakon**//
        \glb many \gl{3}>\gl{3}-kill-\gl{pst}.\gl{pfv} \gl{1+3} monkey//
            \glft ‘We killed many monkeys.’//  
        \endgl 
    \a Panare\\
    \label{pan-2}        \begingl
        \glpreamble Arakon wïyaj ana muku. //
        \gla **arakon** wï-yaj ana **muku**//
        \glb monkey kill-\gl{pst}.\gl{pfv} \gl{1+3} \gl{dist}.\gl{anim}.\gl{vis}//
            \glft ‘We killed that monkey.’//  
        \endgl 
    \a Panare\\
    \label{pan-3}        \begingl
        \glpreamble Wëiki yúchaj Rusiyana kamonton úya onkono. //
        \gla **wëiki** y-u-chaj rusiyana kamonton uya **onkono**//
        \glb deer \gl{lk}-give-\gl{pst}.\gl{pfv} L. \gl{3}\gl{pl} \gl{dat} alive//
            \glft ‘Luciano gave them live deer.’//  
        \endgl 
\xe

\section{Introduction: Panare}

\begin{itemize}
\tightlist
\item
  order is flexible \parencites[128]{payne1993nonconfigurationality}:

  \begin{itemize}
  \tightlist
  \item
    \emph{arakon asa'} / \emph{asa' arakon} `two monkeys'
  \item
    \emph{arakon ta'meñe} / \emph{ta'meñe arakon} `many monkeys'
  \item
    \emph{sïj perikura} `this movie', \emph{kamicha sïj} `this shirt'
  \end{itemize}
\item
  Claim: discontinuous NP = coreferential adjoined nouns \todo{what?}
\item
  Our question: in what sense is this an NP? \todo{who said that?}
\item
  we looked at corpora from five Cariban languages, 1000 records each:

  \begin{itemize}
  \tightlist
  \item
    Tiriyó \parencites{meiraDBtrio}
  \item
    Hixkaryána (\textcites{derbyshire1965textos} via
    \textcites{meiraDBhixka})
  \item
    Ye'kwana \parencites{caceresDByekwana}
  \item
    Akawaio \parencites{caesarDBakawaio}
  \item
    Yawarana \parencites{caceres2020flex}
    \todo{what kind of count do we show here?}
  \end{itemize}
\end{itemize}\begin{tabular}[t]{lllll}

  Language & Words & Nouns & Discontinuous & Pseudo-NPs \\

    Tiriyó &  6641 &  1807 &    12 (0.66%) & 24 (1.33%) \\
Hixkaryána &  4822 &   725 &     2 (0.28%) &  4 (0.55%) \\
   Akawaio &  7394 &  1347 &     7 (0.52%) & 60 (4.45%) \\
  Ye'kwana &  4304 &   829 &    13 (1.57%) & 55 (6.63%) \\
  Yawarana &  4568 &   823 &     9 (1.09%) & 66 (8.02%) \\

\end{tabular}

\todo{Outline of the talk from here}

\section{What is reconstructible?}

\begin{itemize}
\tightlist
\item
  Reconstructible Prior State: adjoined nouns, afterthought
  constructions \todo{did you mean "coreferential" here? }
\item
  very common in Hixkaryána
  \parencites[101-104, 129-135]{hixkaryanaderby1985}
\end{itemize}

\pex\label{}    \a Hixkaryána\\
    \label{derbyshire-1965-p-052-the-origin-of-mawarye-and-woxka-235}        \begingl
        \glpreamble Etenï oske nkeno, Woxka=y, kekonï=hatï, Mawarye, towtï=wya. //
        \gla etenï oske n-ke-no Woxka y 0-ke-konï hatï mawarye t-owtï wya//
        \glb wh.In thus 3Sa-say-Pres:Dbt Woxka Voc 3Sa-say-DPst2 Hrsy Mawarye \gl{3}\gl{r}-sm.sx.sblg Dat//
            \glft ‘"What is (that) saying like that, Woxka," said Mawarye to his brother.’//  
        \endgl 
    \a Hixkaryána\\
    \label{derbyshire-1965-p-061-the-wives-of-mawarye-and-woxka-040}        \begingl
        \glpreamble Ahruthuru yahyekonï, xaryemna=ha. //
        \gla 0-ahru-0-thuru y-ahye-konï xaryemna ha//
        \glb \gl{3}-cover-Nzr-Pst:Pos Rel-shake-DPst2 otter.sp Intens//
            \glft ‘The otter was shaking the stuff covering it.’//  
        \endgl 
    \a Hixkaryána\\
    \label{derbyshire-1965-p-062-the-wives-of-mawarye-and-woxka-071}        \begingl
        \glpreamble Moyorono mokyamo=ha, worïskomo=heno=ha, kekonï=hatï, xaryemna=ha. //
        \gla moyoro-no **mokyamo** ha **worïskomo** heno ha 0-ke-konï hatï xaryemna ha//
        \glb different.place-Nzr 3AnRmCol Intens women Quant Intens 3Sa-say-DPst2 Hrsy otter.sp Intens//
            \glft ‘"They (are) ones far away, the women," said the otter.’//  
        \endgl 
\xe

\begin{itemize}
\item
  Hixkaryana as modern exemplar

  \begin{itemize}
  \tightlist
  \item
    Afterthoughts are really common
  \end{itemize}
\item
  \todo{"Some examples from the other 4 languages that are consistent"}

  \begin{itemize}
  \tightlist
  \item
    Different kinds of discontinuity

    \begin{itemize}
    \tightlist
    \item
      Major class words intervening
    \item
      Phrasal/scope particles intervening (we don't consider this
      discontinuous, as not different from possessive construction)
    \end{itemize}
  \end{itemize}
\end{itemize}

\section{Pattern 1}

\begin{itemize}
\tightlist
\item
  Pattern 1: afterthoughts become less ``separated'' from clause

  \begin{itemize}
  \tightlist
  \item
    Data from Tiriyó showing afterthoughts (apparently w/o pauses)

    \begin{itemize}
    \tightlist
    \item
      Flag future work: checking sound files from Tiriyó
    \end{itemize}
  \item
    This is clearly what has happened in Panare VSO order (cite
    \textcites{payne1994ovs} for frequencies,
    \textcites{gildea2000vpgenesis} for mechanism)
  \end{itemize}
\end{itemize}

\section{Pattern 2}

\begin{itemize}
\tightlist
\item
  Pattern 2: coreferential nouns begin to cohere into something
  resembling continuous NPs

  \begin{itemize}
  \tightlist
  \item
    {[}Dem + N{]} (also {[}N + Dem{]}?)

    \begin{itemize}
    \tightlist
    \item
      Data from Ye'kwana, Yawarana
    \item
      May need to order these depending on what we find
    \end{itemize}

    \ex Yawarana \parencite{caceres2020flex} \\
    \label{ctorat-34}    \begingl
      \glpreamble ëkëtë mërë ëkï //
      \gla ëkëtë **mërë** **ëkï**//
      \glb where \gl{3}\gl{in}:\gl{md} manioc.beer//
       \glft ‘donde está el yaraki?’//  
      \endgl 
    \xe
  \item
    Adding other elements

    \begin{itemize}
    \tightlist
    \item
      Data from Akawaio
    \end{itemize}
  \end{itemize}
\end{itemize}

\section{Discussion}

\begin{itemize}
\tightlist
\item
  Discussion (in place of conclusions)

  \begin{itemize}
  \tightlist
  \item
    Clearly there are discontinuities in the texts, and not all are
    separated by pause/intonation contours
  \item
    Not so clear that they instantiate a theoretical notion of ``NP''
  \item
    Not so clear that even continuous coreferring elements instantiate a
    theoretical notion of NP
  \item
    But one seems to be developing

    \begin{itemize}
    \tightlist
    \item
      {[}DEM N{]}
    \item
      more elaborated in Akawaio (presumably influenced by English)
    \end{itemize}
  \end{itemize}
\end{itemize}

\printbibliography

\end{document}