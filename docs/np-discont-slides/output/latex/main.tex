\documentclass[10pt]{article}
\usepackage[
    landscape,
    twocolumn,
    a4paper,
    left=0.7in,
    right=0.7in,
    bottom=1in,
    top=0.7in]{geometry}

\input{preamble.tex}
\addbibresource{/home/florianm/Dropbox/research/cariban/papers/np_discont/data/cldf/sources.bib}
\lingset{belowglpreambleskip=-1ex,everyglpreamble=\itshape,aboveglftskip=-0.5ex,aboveexskip=-0.7ex,belowexskip=-1ex}

\newGlossingAbbrev{3}{third person}
\newGlossingAbbrev{n}{noun}
\newGlossingAbbrev{np}{noun phrase}
\newGlossingAbbrev{1+2}{1+2}
\newGlossingAbbrev{a}{a}
\newGlossingAbbrev{p}{p}
\newGlossingAbbrev{t}{t}
\newGlossingAbbrev{g}{g}
\newGlossingAbbrev{o}{o}
\newGlossingAbbrev{dem}{dem}
\newGlossingAbbrev{2}{2}
\newGlossingAbbrev{1}{1}

\setmainfont{Linux Libertine O}

\begin{document}
\begin{center}
\Large \bfseries {Apparent discontinuous NPs in the Cariban family from the perspective of diachronic typology}
\end{center}
%\tableofcontents

\section{Introduction: Panare}

\begin{itemize}
\tightlist
\item
  \textcites{payne1993nonconfigurationality} discusses discontinuous
  nominal expressions in Panare:
\end{itemize}

\pex\label{}    \a Panare\\
    \label{pan-0}        \begingl
        \glpreamble Kana tënyaj chu tityasa’. //
        \gla **kana** t-ën-yaj chu **tityasa’**//
        \glb fish \gl{1}>\gl{3}-catch-\gl{pst}.\gl{pfv} \gl{1}\gl{sg} one//
            \glft ‘I caught one fish.’//  
        \endgl 
    \a Panare\\
    \label{pan-1}        \begingl
        \glpreamble Ta’meñe nuwïyaj ana arakon. //
        \gla **ta’meñe** nu-wï-yaj ana **arakon**//
        \glb many \gl{3}>\gl{3}-kill-\gl{pst}.\gl{pfv} \gl{1+3} monkey//
            \glft ‘We killed many monkeys.’//  
        \endgl 
    \a Panare\\
    \label{pan-2}        \begingl
        \glpreamble Arakon wïyaj ana muku. //
        \gla **arakon** wï-yaj ana **muku**//
        \glb monkey kill-\gl{pst}.\gl{pfv} \gl{1+3} \gl{dist}.\gl{anim}.\gl{vis}//
            \glft ‘We killed that monkey.’//  
        \endgl 
    \a Panare\\
    \label{pan-3}        \begingl
        \glpreamble Wëiki yúchaj Rusiyana kamonton úya onkono. //
        \gla **wëiki** y-u-chaj rusiyana kamonton uya **onkono**//
        \glb deer \gl{lk}-give-\gl{pst}.\gl{pfv} L. \gl{3}\gl{pl} \gl{dat} alive//
            \glft ‘Luciano gave them live deer.’//  
        \endgl 
\xe

\section{Adnominal modification in the Cariban family}

\begin{itemize}
\tightlist
\item
  possible combinations:

  \begin{itemize}
  \tightlist
  \item
    N\textsubscript{\gl{mod}} + N
  \item
    ADV + N
  \item
    DEM + N
  \end{itemize}
\item
  all modifying elements can also occur without a noun

  \begin{itemize}
  \tightlist
  \item
    demonstratives form standalone NPs
  \end{itemize}
\item
  property concepts are either expressed by nouns (`the big one') or
  adverbially \parencites{meira2009property}
\end{itemize}

\todo{I changed the order of this slide and the next, what do you think?}

\section{Adnominal modification in Panare}

\begin{itemize}
\tightlist
\item
  combinations attested as discontinuous can occur adjacently, too
\item
  our question: in what sense can these be considered constituents?
  \todo{I really don't know how to talk about phrasal and second-position particles here?}

  \begin{itemize}
  \tightlist
  \item
    flexible order \parencites[128]{payne1993nonconfigurationality}:

    \begin{itemize}
    \tightlist
    \item
      \emph{arakon asa'} / \emph{asa' arakon} `two monkeys'
    \item
      \emph{arakon ta'meñe} / \emph{ta'meñe arakon} `many monkeys'
    \item
      \emph{sïj perikura} `this movie', \emph{kamicha sïj} `this shirt'
    \end{itemize}
  \item
    no salient structure, apart from modifying and referential semantics
  \end{itemize}
\item
  contrast with ``genitive'' NPs:

  \begin{itemize}
  \tightlist
  \item
    fixed possessor-possessum order
  \item
    linking prefix on possessum
  \item
    complementary distribution of lexical possessor and third person
    prefix
  \item
    prosodic integration
  \end{itemize}
\end{itemize}

\pex\label{nps} \a\label{panposs} \emph{Toman y-árako} `Tom's hat'
\parencites[125]{payne1993nonconfigurationality} \a\label{araposs}
\emph{yï-matan} `his/her shoulder', \emph{Toman mátan} `Tom's shoulder'
\parencites[76]{panarepayne2013} \xe

\begin{itemize}
\tightlist
\item
  possessive NP (along with PP and transitive VP) reconstructible to
  Proto-Cariban
\end{itemize}

\section{Our study}

\begin{itemize}
\tightlist
\item
  we searched modified nouns in corpora from five Cariban languages,
  each 1000 records:

  \begin{itemize}
  \tightlist
  \item
    Tiriyó \parencites{meiraDBtrio}
  \item
    Hixkaryána (\textcites{derbyshire1965textos} via
    \textcites{meiraDBhixka})
  \item
    Ye'kwana \parencites{caceresDByekwana}
  \item
    Akawaio \parencites{akawaiocaesar2003}
  \item
    Yawarana \parencites{caceres2020flex}
  \end{itemize}
\item
  first finding: these are rare in general
\end{itemize}\begin{tabular}[t]{lllll}

  Language & Words & Nouns & Modified nouns & Discontinuous \\

    Tiriyó &  6641 &  1807 &     27 (1.49%) &     8 (0.44%) \\
Hixkaryána &  4822 &   725 &      4 (0.55%) &     0 (0.00%) \\
   Akawaio &  7394 &  1347 &     67 (4.97%) &     4 (0.30%) \\
  Ye'kwana &  4297 &   828 &     56 (6.76%) &     8 (0.97%) \\
  Yawarana &  4568 &   823 &     74 (8.99%) &     4 (0.49%) \\

\end{tabular}

\section{Investigated languages}

\includegraphics{figures/map.svg}

\begin{itemize}
\tightlist
\item
  convenience sample
\item
  5 languages with annotated corpora, first sounding
\item
  note: Venezuelan branch is not solidly established
\end{itemize}

\section{Nouns with modifiers}\begin{tabular}[t]{lllllll}

           & ADV+N & DEM+DEM & DEM+N & N+N & NUM+N & Nmod+N \\

   Akawaio &     4 &       0 &    43 &   1 &    11 &      8 \\
Hixkaryána &     0 &       0 &     3 &   1 &     0 &      0 \\
    Tiriyó &     0 &       0 &    17 &   2 &     3 &      5 \\
  Yawarana &     1 &       0 &    48 &   2 &     8 &     15 \\
  Ye'kwana &     1 &       2 &    46 &   1 &     3 &      3 \\

\end{tabular}

\begin{enumerate}
\def\labelenumi{\arabic{enumi}.}
\tightlist
\item
  very rare in Hixkaryána
\item
  salient pattern: DEM + N in the three Venezuelan languages
\end{enumerate}

\section{Nouns with modifiers: demonstratives}

\begin{itemize}
\tightlist
\item
  {[}DEM N{]}, but also {[}N DEM{]}
  \todo{I will insert some examples here for both orders}
\end{itemize}\begin{tabular}[t]{llllll}

      & Akawaio & Hixkaryána & Tiriyó & Yawarana & Ye'kwana \\

DEM N &      37 &          1 &      7 &       35 &       37 \\
N DEM &       2 &          1 &      1 &        9 &        3 \\

\end{tabular}

\begin{itemize}
\tightlist
\item
  as in Panare, order is not absolute
\item
  not in favor of constituency
\end{itemize}

\section{Nouns with modifiers: nouns with other modifiers}

\begin{table}
\caption{Modifying numerals}
\label{tab:nummod}
\centering
\begin{tabular}{lllll}
\toprule
        & Akawaio & Tiriyó & Yawarana & Ye'kwana \\
\midrule
  N NUM &       1 &      0 &        2 &        0 \\
N NUM N &       1 &      0 &        0 &        0 \\
  NUM N &       9 &      3 &        6 &        3 \\
\bottomrule
\end{tabular}

\end{table}

\begin{table}
\caption{Other modifying adverbs}
\label{tab:advmod}
\centering
\begin{tabular}{llll}
\toprule
              & Akawaio & Yawarana & Ye'kwana \\
\midrule
        ADV N &       0 &        1 &        1 \\
        N ADV &       1 &        0 &        0 \\
    NUM ADV N &       2 &        0 &        0 \\
NUM NUM ADV N &       1 &        0 &        0 \\
\bottomrule
\end{tabular}

\end{table}

\begin{table}
\caption{Modifying nouns}
\label{tab:nmod}
\centering
\begin{tabular}{lllll}
\toprule
            & Akawaio & Tiriyó & Yawarana & Ye'kwana \\
\midrule
     N Nmod &       1 &      3 &        8 &        0 \\
     Nmod N &       7 &      2 &        6 &        3 \\
Nmod N Nmod &       0 &      0 &        1 &        0 \\
\bottomrule
\end{tabular}

\end{table}

\todo{I will insert some examples on a separate slide}
\todo{what is our take away message here, apart from no fixed order?}

\section{Modified nouns as arguments of postpositions}

\ex Tiriyó \parencite{meiraDBtrio} \\
\label{dados-09-kereramake-p-041-270}    \begingl
    \glpreamble (Ape) sen=nai... tuna mono juuwë iitën=mahtao, nemanakan tïïkae. //
    \gla Ape sen n-ai **tuna** **mono** juuwë i-V:-të-0-n mahtao n-e-manaka-0-n tï-V:-ka-e//
    \glb Ape 3InPx 3Sa-Cop water big on \gl{3}-Sa-go-\gl{g}-? when 3Sa-Detr-roll-Pres-Dbt Pst-Sa-say-Pst//
        \glft ‘(Ape) 'This one is... when it's on a big river, it turns,' (she) said.’//  
    \endgl 
\xe

\ex Tiriyó \parencite{meiraDBtrio} \\
\label{mini-disc-lsmtam13set0001-marciano-little-introduction-005}    \begingl
    \glpreamble Serë=pë panpira=pë anja=nai oroko=me. Meta? Ir=apo. //
    \gla **serë** pë **panpira** pë anja n-ai oroko me m-eta-0 irë apo//
    \glb 3InPx about book about \gl{1+3} 3Sa-Cop work Ess \gl{2}\gl{a}-hear-\gl{i}.Pst 3InAna like//
        \glft ‘We're now working about this, about this book/paper (= i.e. we're writing something). Did you hear? That's it.  ’//  
    \endgl 
\xe

\section{Modified nouns as arguments of postpositions}

\begin{table}
\caption{Modified nouns as arguments of postpositions}
\label{tab:postparg}
\centering
\begin{tabular}{lll}
\toprule
         & Single postposition & Two postpositions \\
\midrule
 Akawaio &                   9 &                 1 \\
  Tiriyó &                   8 &                 3 \\
Yawarana &                  10 &                 0 \\
Ye'kwana &                   3 &                 1 \\
\bottomrule
\end{tabular}

\end{table}

\section{Modified nouns as arguments of transitive verbs}

\ex Hixkaryána \parencite[60]{derbyshire1965textos} \\
\label{derbyshire-1965-p-060-the-wives-of-mawarye-and-woxka-005}    \begingl
    \glpreamble Horyetho monï yahuryatxkonï. //
    \gla **horye-tho** **monï** y-ahur-yatxkonï//
    \glb big.one-Pst 3InRm Rel-shut.in-DPst2:Col//
        \glft ‘They sealed off that one, the big one.’//  
    \endgl 
\xe

\ex Akawaio \parencite{akawaiocaesar2003} \\
\label{pingkas-personal-narrative-005-interviewer}    \begingl
    \glpreamble Mörabai ji azaurogï'pï dïboi mïgï pandong egamaba aza'rörö, //
    \gla Mörö abai ji a-saurogï-'pï tïbo-i mïgï **pandong** egama-ba **aza'rö-rö**//
    \glb that from Emph \gl{2}-talk-Past after-Psd Hes story tell-in.order.to two-Emph//
        \glft ‘Then after you talk, then you can tell a story maybe two’//  
    \endgl 
\xe

\todo{this is the only [N Vt N] example! 1. what is the -ba in Akawaio? 2. I will search the audio for this example...}

\section{Modified nouns as arguments of transitive verbs}\begin{tabular}[t]{lll}

           & Interrupted by verb & Preceding verb \\

   Akawaio &                   1 &             17 \\
Hixkaryána &                   0 &              1 \\
  Yawarana &                   0 &              4 \\
  Ye'kwana &                   0 &              1 \\

\end{tabular}

\section{Modified nouns as possessors: bracketing paradox}

\pex\label{}    \a Akawaio\\
    \label{ra-eagle-story-034}        \begingl
        \glpreamble Eneboro'pï se wïk ezek Eneboro'pï //
        \gla Eneboro'pï **se** **wïk** **ezek** Eneboro'pï//
        \glb Eneboro'pï this mountain name Eneboro'pï//
            \glft ‘The name of this mountain was Eneboro'pï’//  
        \endgl 
    \a Ye'kwana\\
    \label{ctoabjpic-027}        \begingl
        \glpreamble Tüw- tüwü wanö etü küna'jaakö yaawö, Tajööju. //
        \gla **…-tüwü** **wanö** **e-ötü-0** kün-a'ja-aakö yaawö Tajööju//
        \glb \gl{…}-\gl{3}.\gl{sg} abeja \gl{rel}-nombre-\gl{poss}.\gl{3} \gl{3}.\gl{pas}-\gl{cop}-\gl{pdi} entonces Tajööju//
            \glft ‘El nombre de la abeja era Tajööju.’//  
        \endgl 
    \a Yawarana\\
    \label{ctorosq-66}        \begingl
        \glpreamble tëwï wurijyan inwa yëjnë tapiremï //
        \gla **tëwï** **wurijyan** **inwa-Ø** y-ëjnë tapire-mï//
        \glb \gl{3}\gl{sg} woman glute-\gl{poss} \gl{rel}-under red-\gl{nzr}//
            \glft ‘debajo de las nalgas de la mujer estaba rojo’//  
        \endgl 
\xe

\section{Discontinuity: intervening particles}

\todo{I will show **some** intervening phrasal particles, then show same particles with bonafide NPs}

\section{Discontinuity: intervening intransitive verbs}

\ex Tiriyó \parencite{meiraDBtrio} \\
\label{data-01-yakari-01-p-127-022}    \begingl
    \glpreamble Ma... Mono weine kïrïmuku irënehka. //
    \gla ma **mono** w-ei-ne **kïrïmuku** irënehka//
    \glb Attn big 1Sa-Cop-DPst young.man finally//
        \glft ‘Well... I finally grew up, I became a young man.  ’//  
    \endgl 
\xe

\ex Ye'kwana \parencite{caceresDByekwana} \\
\label{ctoabjpic-094}    \begingl
    \glpreamble Ee, wodije tüwü küna'jaakö wanö. //
    \gla ee wodi=je **tüwü** kün-a'ja-aakö **wanö**//
    \glb sí mujer=\gl{atrb} \gl{3}.\gl{sg} \gl{3}.\gl{pas}-\gl{cop}-\gl{pdi} abeja//
        \glft ‘Sí, ella la abeja era una mujer’//  
    \endgl 
\xe

\begin{itemize}
\tightlist
\item
  only 5 examples
\end{itemize}

\section{Discontinuity: intervening adverbs}

\ex Ye'kwana \parencite{caceresDByekwana} \\
\label{convchur-009}    \begingl
    \glpreamble aneija mödöje yadanawi chü'tajötüdü na, yadanawi mödöjemmödö //
    \gla **aneija** mödöje **yadanawi** i-tü'tajötü-dü na yadanawi mödöje mödö//
    \glb otro así no\_indígena \gl{intr}.\gl{c}-pensar-\gl{nzr}.\gl{1} \gl{3}.\gl{cop} no\_indígena así DEM2in//
        \glft ‘otros criollos piensan así, el criollo es así’//  
    \endgl 
\xe

\begin{itemize}
\tightlist
\item
  only 2 examples
\end{itemize}

\section{Discontinuity: A noun}

\ex Ye'kwana \parencite{caceresDByekwana} \\
\label{ctoabjpic-008}    \begingl
    \glpreamble aakömma wanö... kiyede munu kunu'kwai. //
    \gla **aakö-mma** wanö kiyede **munu-0** kün-u'ka-i//
    \glb dos-\gl{excl} abeja yuca tubérculo-\gl{poss}.\gl{3} \gl{3}/\gl{3}.\gl{pas}-sacar-\gl{pas}//
        \glft ‘La abeja sacó dos tuberculos de yuca.’//  
    \endgl 
\xe

\section{Diachrony: afterthoughts}

\begin{itemize}
\tightlist
\item
  starting point are afterthought constructions
\item
  very common still in Hixkaryána
  \parencites[101-104, 129-135]{hixkaryanaderby1985}
  \todo{I will add highlights in the orthographic line on this and the next slides -- so people see the commas}
\end{itemize}

\pex\label{}    \a Hixkaryána\\
    \label{derbyshire-1965-p-040-the-buzzard-people-and-their-servant-the-sloth-109}        \begingl
        \glpreamble Onï=ha, tïmko=ha, oyeheka=wya=ha, kekonï=hatï. //
        \gla onï ha t-ïm-ko ha oy-eheka wya ha 0-ke-konï hatï//
        \glb 3InPx Intens Gen-give-Imper Intens \gl{2}-brother Dat Intens 3Sa-say-DPst2 Hrsy//
            \glft ‘"Give this to your brother," he said.’//  
        \endgl 
    \a Hixkaryána\\
    \label{derbyshire-1965-p-061-the-wives-of-mawarye-and-woxka-040}        \begingl
        \glpreamble Ahruthuru yahyekonï, xaryemna=ha. //
        \gla 0-ahru-0-thuru y-ahye-konï xaryemna ha//
        \glb \gl{3}-cover-Nzr-Pst:Pos Rel-shake-DPst2 otter.sp Intens//
            \glft ‘The otter was shaking the stuff covering it.’//  
        \endgl 
    \a Hixkaryána\\
    \label{derbyshire-1965-p-062-the-wives-of-mawarye-and-woxka-071}        \begingl
        \glpreamble Moyorono mokyamo=ha, worïskomo=heno=ha, kekonï=hatï, xaryemna=ha. //
        \gla moyoro-no **mokyamo** ha **worïskomo** heno ha 0-ke-konï hatï xaryemna ha//
        \glb different.place-Nzr 3AnRmCol Intens women Quant Intens 3Sa-say-DPst2 Hrsy otter.sp Intens//
            \glft ‘"They (are) ones far away, the women," said the otter.’//  
        \endgl 
\xe

\section{Diachrony: afterthoughts}

\ex Ye'kwana \parencite{caceresDByekwana} \\
\label{ctoabjpic-003}    \begingl
    \glpreamble Yo'jüdüje küna'jaakö tüwü, akudi... //
    \gla i-wo'jüdü-0=je kün-a'ja-aakö **tüwü** **akudi**//
    \glb \gl{3}-suegra-\gl{poss}.\gl{3}=\gl{atrb} \gl{3}\gl{s}.\gl{pas}-\gl{cop}-\gl{pdi} \gl{3}.\gl{sg} picure//
        \glft ‘La picure era como su suegra.’//  
    \endgl 
\xe

\ex Tiriyó \parencite{meiraDBtrio} \\
\label{dados-09-kereramake-p-047-374}    \begingl
    \glpreamble (Ape) Irë=mao tënepiitae, mëërë=ja turëe, werekeru=ja. //
    \gla Ape irë mao t-ënepiita-e **mëërë** ja t-urë-e **werekeru** ja//
    \glb Ape 3InAna at.time Pst-be.bored-Pst 3AnMd Agt Pst-talk.to-Pst parakeet.sp Agt//
        \glft ‘(Ape) Then he became tired of it, and that one talked to him, werekeru.’//  
    \endgl 
\xe

\ex Yawarana \parencite{caceres2020flex} \\
\label{ctowaru-19}    \begingl
    \glpreamble warë imu yakërë ma wejsaj ti ta, imu warë imukuru, yatanë //
    \gla warë imu y-akërë ma wej-saj ti ta imu warë **i-muku-ru** **yatanë**//
    \glb thus \gl{3}:father:\gl{pos} \gl{rel}-\gl{com} \gl{restr} \gl{cop}-\gl{perf} like like \gl{3}:father:\gl{pos} thus \gl{3}-child-\gl{poss} young//
        \glft ‘así con su papá solamente, el papá y el hijo, un muchacho’//  
    \endgl 
\xe

\section{Diachrony: prosodically integrated afterthoughts}

\begin{itemize}
\tightlist
\item
  \textbf{afterthoughts become less ``separated'' from clause \&
  prosodically integrated}
\end{itemize}

\ex Hixkaryána \parencite[19]{derbyshire1965textos} \\
\label{derbyshire-1965-p-019-the-origin-of-night-and-darkness-081}    \begingl
    \glpreamble Amna nawanyeko=ha, mosonï hanahtorï=wya, ketxkonï=hatï, kekonï=hatï noro horykomo. //
    \gla amna n-awanye-ko ha mosonï 0-hana-hto-rï wya 0-ke-txkonï hatï 0-ke-konï hatï **noro** **horykomo**//
    \glb \gl{1+3} 3Sa-get.dark-RPst1 Intens 3AnPx Rel-ear-Neg.Act-Pos Agt 3Sa-say-DPst2:Col Hrsy 3Sa-say-DPst2 Hrsy 3AnAna chief.man//
        \glft ‘"We were in darkness, through the foolishness of this one," they said, he, the chief man, said.’//  
    \endgl 
\xe

\ex Yawarana \parencite{caceres2020flex} \\
\label{ctowaru-36}    \begingl
    \glpreamble tajne ma ana wësarë ta ti waraijto pana yapijtom pana //
    \gla tajne ma ana wë-sarë ta-Ø ti **waraijto-Ø** pana **yapijtom** pana//
    \glb \gl{3}\gl{pl} \gl{restr} \gl{1+3} kill-\gl{inm} say-\gl{ipfv} like husband-\gl{poss} \gl{adrs} old \gl{adrs}//
        \glft ‘ellos nos están matando, le dijo a su esposo, a su viejo’//  
    \endgl 
\xe

\section{Diachrony: prosodically integrated afterthoughts}

\pex\label{}    \a Tiriyó\\
    \label{mini-disc-lsmtam13set0001-marciano-little-introduction-003}        \begingl
        \glpreamble Ma, jijomii=tae=rë jeka tarëno=tae nai, Minausi , irë apo. Meta? //
        \gla ma **ji-jomi-V:** tae rë j-eka-0 **tarëno** tae n-ai Minausi irë apo m-eta-0//
        \glb Attn \gl{1}-language-Psfx by Idtf \gl{1}-name-Psfx Tiriyó by 3Sa-Cop Minausi 3InAna like \gl{2}\gl{a}-hear-\gl{i}.Pst//
            \glft ‘Well, in my own language my name, in (the language of the) Tiriyó, it is, Minausi, like this. OK? (= Did you hear?).  ’//  
        \endgl 
    \a Tiriyó\\
    \label{minidisc-xxx-torohpe-iwehtoponpe-002}        \begingl
        \glpreamble Ma, irëme wëene, ipërih=tae anja eekuhpë, ahn..., atï irë, Kuruni. //
        \gla ma irëme w-ëe-ne **ipërih** tae **anja** **0-eeku-hpë** … atï irë Kuruni//
        \glb Attn so 1Sa-come-DPst creek along \gl{1+3} \gl{3}-stream-\gl{p}.pst \gl{…} wh.In 3InAna Kuruni//
            \glft ‘Well, then I came, along the creeks/tribuaries, our old river... what-do-you-call-it, the Kuruni (Coeroeni).  ’//  
        \endgl 
\xe

\section{Diachrony: prosodically integrated afterthoughts}

\ex Yawarana \parencite{caceres2020flex} \\
\label{histgrme-134}    \begingl
    \glpreamble yaijkotapïti wejsapë wïrë yïkïnï parë weroro //
    \gla yaijkota-pïti-Ø wej-sapë wïrë **y-ïkïnï** parë **weroro**//
    \glb bark-\gl{plac}-\gl{ipfv} \gl{cop}-\gl{perf} \gl{1}\gl{sg} \gl{rel}-pet and dog//
        \glft ‘mis perros también estaban [[[ladrando]]]’//  
    \endgl 
\xe

\begin{itemize}
\tightlist
\item
  clearly what has happened to yield innovative Panare VAP order
  \parencites[595]{payne1994ovs}{gildea2000vpgenesis}
\end{itemize}

\begin{table}
\caption{Pattern frequencies in Panare past-perfective clauses}
\label{tab:panarefreq}
\centering
\begin{tabular}{ll}
\toprule
    Order &   Frequency \\
\midrule
p-V (A) P & 27 (56.25%) \\
      p-V & 19 (39.58%) \\
      P V &   2 (4.17%) \\
\bottomrule
\end{tabular}

\end{table}

\section{Discussion}

\begin{itemize}
\tightlist
\item
  apparent discontinuous noun phrases are not constituents

  \begin{itemize}
  \tightlist
  \item
    they are prosodically integrated afterthoughts
  \item
    usually not translated as afterthought (!)
  \end{itemize}
\item
  no evidence that even continuous coreferential elements are
  constituents

  \begin{itemize}
  \tightlist
  \item
    contrast with bona fide possessive NPs
  \item
    possible exception: incipient {[}DEM N{]} in Akawaio, Ye'kwana,
    Yawarana
  \end{itemize}
\item
  raises questions about discontinuous constituents in general
\end{itemize}

\printbibliography

\end{document}